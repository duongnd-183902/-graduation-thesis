\documentclass[a4paper,13pt,3p]{report}
 \usepackage{scrextend}
\changefontsizes{13pt}
\usepackage[utf8]{vietnam}
% \usepackage[utf8]{inputenc}
\usepackage[top=2cm, bottom=2cm, left=3.5cm, right=2.5cm]{geometry}
\usepackage{xurl}
\usepackage{appendix}
\usepackage{babel}
\usepackage{xcolor}
\usepackage{outlines}
\usepackage{graphicx} % Cho phép chèn hỉnh ảnh
\usepackage{fancybox} % Tạo khung box
\usepackage{indentfirst} % Thụt đầu dòng ở dòng đầu tiên trong đoạn
\usepackage{amsthm} % Cho phép thêm các môi trường định nghĩa
\usepackage{latexsym} % Các kí hiệu toán học
\usepackage{amsmath} % Hỗ trợ một số biểu thức toán học
\usepackage{amssymb} % Bổ sung thêm kí hiệu về toán học
\usepackage{amsbsy} % Hỗ trợ các kí hiệu in đậm
\usepackage{times} % Chọn font Time New Romans
\usepackage{array} % Tạo bảng array
\usepackage{enumitem} % Cho phép thay đổi kí hiệu của list
\usepackage{subfiles} % Chèn các file nhỏ, giúp chia các chapter ra nhiều file hơn
\usepackage{titlesec} % Giúp chỉnh sửa các tiêu đề, đề mục như chương, phần,..
\usepackage{titletoc}
\usepackage{chngcntr} % Dùng để thiết lập lại cách đánh số caption,..
\usepackage{pdflscape} % Đưa các bảng có kích thước đặt theo chiều ngang giấy
\usepackage{afterpage}
\usepackage[ruled,vlined]{algorithm2e}  % Hỗ trợ viết các giải thuật
\usepackage{capt-of} % Cho phép sử dụng caption lớn đối với landscape page
\usepackage{multirow} % Merge cells
\usepackage{fancyhdr} % Cho phép tùy biến header và footer
% \usepackage[natbib,backend=biber,style=ieee]{biblatex} % Giúp chèn tài liệu tham khảo

\usepackage[font=small,labelfont=bf]{caption}

\usepackage{listings}
\usepackage{float}
\usepackage{subcaption}

\usepackage[nonumberlist, nopostdot, nogroupskip, acronym]{glossaries}
\usepackage{glossary-superragged}
\setglossarystyle{superraggedheaderborder}
\usepackage{setspace}
\usepackage{parskip}

% package content table
\usepackage{tocbasic}

\usepackage{blindtext}


% ===================================================

% \renewcommand{\bibname}{Danh_sach_tai_lieu_tham_khao} 
\usepackage[backend=bibtex,style=ieee]{biblatex}  %backend=biber is 'better'

\usepackage{hyperref}

\usepackage{tabularx}
% \usepackage{caption}
% \restylefloat{table}
% \usepackage{longtable}
%==================================%
\addbibresource{reference.bib} % chèn file chứa danh mục tài liệu tham khảo vào 

\include{lstlisting} % Phần này cho phép chèn code và formatting code như C, C++, Python

%\makeglossaries
\makenoidxglossaries

% Danh mục thuật ngữ và từ viết tắt


\newglossaryentry{VRF}{
    type=\acronymtype,
    name={VRF},
    description={Hàm sinh số ngẫu nhiên có thể kiểm chứng}
}

\newglossaryentry{Transaction}{
    type=\acronymtype,
    name={Transaction},
    description={Hành động được thực hiện bởi tài khoản của người dùng không phải của hợp đồng thông minh}
}

\newglossaryentry{Block}{
    type=\acronymtype,
    name={Block},
    description={Một lô các transaction cùng với blockhash của block trước đó}
}
\newglossaryentry{Blockhash}{
    type=\acronymtype,
    name={Blockhash},
    description={Mã băm của block thỏa mãn điều kiện độ khó của thuật toán bằng chứng công việc (PoW)}
}

\newglossaryentry{EVM}{
    type=\acronymtype,
    name={EVM},
    description={Máy ảo ethereum (Ethereum virtual machine)}
}

\newglossaryentry{Blockchain}{
    type=\acronymtype,
    name={Blockchain},
    description={Chuỗi khối - Cơ sở dữ liệu công khai, phân tán}
}
\newglossaryentry{Gas}{
    type=\acronymtype,
    name={Gas},
    description={Thước đo về số bước tính toán của EVM}
}

\newglossaryentry{Oracle}{
    type=\acronymtype,
    name={Oracle},
    description={Thực thể kết nối blockchain và các hệ thống bên ngoài}
}
\newglossaryentry{Precompile}{
    type=\acronymtype,
    name={Precompile},
    description={Đoạn code sẽ được biên dịch tính toán trước và được sử dụng như đầu vào cho chương trình}
}

\newglossaryentry{dApps}{
    type=\acronymtype,
    name={dApps},
    description={Các ứng dụng phi tập trung}
}

% ===================================================


\fancypagestyle{plain}{%
\fancyhf{} % clear all header and footer fields
\fancyfoot[RO,RE]{\thepage} %RO=right odd, RE=right even
\renewcommand{\headrulewidth}{0pt}
\renewcommand{\footrulewidth}{0pt}}

\setlength{\headheight}{10pt}

\def \TITLE{GRADUATION THESIS}
\def \AUTHOR{Trần Văn A}

% ===================================================
\titleformat{\chapter}[hang]{\centering\bfseries}{CHAPTER \thechapter.\ }{0pt}{}[]

\titleformat 
    {\chapter} % command
    [hang] % shape
    {\centering\bfseries} % format
    {CHƯƠNG \thechapter.\ } % label
    {0pt} %sep
    {} % before
    [] % after
\titlespacing*{\chapter}{0pt}{-20pt}{20pt}



\titleformat
    {\section} % command
    [hang] % shape
    {\bfseries} % format
    {\thechapter.\arabic{section}\ \ \ \ } % label
    {0pt} %sep
    {} % before
    [] % after
\titlespacing{\section}{0pt}{\parskip}{0.5\parskip}

\titleformat
    {\subsection} % command
    [hang] % shape
    {\bfseries} % format
    {\thechapter.\arabic{section}.\arabic{subsection}\ \ \ \ } % label
    {0pt} %sep
    {} % before
    [] % after
\titlespacing{\subsection}{0pt}{\parskip}{0.5\parskip}

\renewcommand\thesubsubsection{\alph{subsubsection}}
\titleformat
    {\subsubsection} % command
    [hang] % shape
    {\bfseries} % format
    {\alph{subsubsection}, \ } % label
    {0pt} %sep
    {} % before
    [] % after
\titlespacing{\subsubsection}{50pt}{\parskip}{0.5\parskip}

% ===================================================

\hypersetup{pdfborder = {0 0 0}}
\hypersetup{pdftitle={\TITLE},
	pdfauthor={\AUTHOR}}
	
\usepackage[all]{hypcap} % Cho phép tham chiếu chính xác đến hình ảnh và bảng biểu

\graphicspath{{figures/}{../figures/}} % Thư mục chứa các hình ảnh

\counterwithin{figure}{chapter} % Đánh số hình ảnh kèm theo chapter. Ví dụ: Hình 1.1, 1.2,..

\title{\bf \TITLE}
\author{\AUTHOR}

\setcounter{secnumdepth}{3} % Cho phép subsubsection trong report
% \setcounter{tocdepth}{3} % Chèn subsubsection vào bảng mục lục

\theoremstyle{definition}
\newtheorem{definition}{Definition}[section]

\theoremstyle{theorem}
\newtheorem{theorem}{Định lý}[section]
\newtheorem{cy}[theorem]{Chú ý}%[subsection]
\newtheorem{bd}[theorem]{Bổ đề}%[subsection]
\newtheorem{bt}[theorem]{Bài toán}%[subsection]
\newtheorem{md}[theorem]{Mệnh đề}%[subsection]
\newtheorem{Bt}[theorem]{Bài toán}%[subsection]
\newtheorem{hq}[theorem]{Hệ quả}%[subsection]
\newtheorem{kh}[theorem]{Kí hiệu}%[subsection]
\newtheorem{vd}[theorem]{\bf Ví dụ}%[subsection]
\newtheorem{nx}[theorem]{\bf Nhận xét}%[subsection]

% 1 
\newtheorem{theorem1}{cac' Định nghia}[section]
% 2
\newtheorem{dn}[theorem1]{\bf Định nghĩa}%[subsection] 
\newtheorem{dl}[theorem]{\bf Định lý}
\newcommand\tab[1][0.7cm]{\hspace*{#1}}
\onehalfspacing
\setlength{\parskip}{6pt}
\setlength{\parindent}{15pt}





% =========================== BODY ===============
\begin{document}
% \newgeometry{top=2cm, bottom=2cm, left=2cm, right=2cm}
\subfile{Cover} % Phần bìa
% \restoregeometry

% ===================================================
\pagestyle{empty} % Header và footer rỗng
%\newpage
%\pagenumbering{gobble} % Xóa page numbering ở cuối trang
%\subfile{chapters/0_1_subject.tex}

% \pagestyle{empty} % Header và footer rỗng
\newpage
\pagenumbering{gobble} % Xóa page numbering ở cuối trang
\subfile{Chapter/0_2_Acknowledgment.tex}

% \pagestyle{empty} % Header và footer rỗng
\newpage
\pagenumbering{gobble} % Xóa page numbering ở cuối trang
\subfile{Chapter/0_3_Abstract.tex}


% ===================================================
% \pagestyle{empty} % Header và footer rỗng
\newpage
\pagenumbering{gobble} % Xóa page numbering ở cuối trang
\renewcommand*\contentsname{MỤC LỤC}
\titlecontents{chapter}
    [0.0cm]             % left margin
    {\bfseries\vspace{0.3cm}}                  % above code
    {{\bfseries{\scshape} CHƯƠNG \thecontentslabel.\ }} % numbered format
    {}         % unnumbered format
    {\titlerule*[0.3pc]{.}\contentspage}         % filler-page-format, e.g dots
    
\titlecontents{section}
    [0.0cm]             % left margin
    {\vspace{0.3cm}}                  % above code
    {\thecontentslabel \ } % numbered format
    {}         % unnumbered format
    {\titlerule*[0.3pc]{.}\contentspage}         % filler-page-format, e.g dots
    
\titlecontents{subsection}
    [1.0cm]             % left margin
    {\vspace{0.3cm}}                  % above code
    {\thecontentslabel \ } % numbered format
    {}         % unnumbered format
    {\titlerule*[0.3pc]{.}\contentspage}         % filler-page-format, e.g dots

 % Tạo mục lục tự động
\addtocontents{toc}{\protect\thispagestyle{empty}}
\tableofcontents 
\thispagestyle{empty}
\cleardoublepage

\pagenumbering{roman}
%Tạo danh mục hình vẽ.
\renewcommand{\listfigurename}{DANH MỤC HÌNH ẢNH}
{\let\oldnumberline\numberline
\renewcommand{\numberline}{Hình~\oldnumberline}
\listoffigures} 
% \phantomsection\addcontentsline{toc}{section}{\numberline {} DANH MỤC HÌNH VẼ}
\newpage


 %Tạo danh mục bảng biểu.
\renewcommand{\listtablename}{DANH MỤC BẢNG BIỂU}
{\let\oldnumberline\numberline
\renewcommand{\numberline}{Bảng~\oldnumberline}
\listoftables}
% \phantomsection\addcontentsline{toc}{section}{\numberline {} DANH MỤC BẢNG BIỂU}

\glsaddall 
% \renewcommand*{\glossaryname}{Danh sách thuật ngữ}
\renewcommand*{\acronymname}{DANH MỤC THUẬT NGỮ VÀ VIẾT TẮT}
\renewcommand*{\entryname}{Thuật ngữ}
\renewcommand*{\descriptionname}{Ý nghĩa}
\printnoidxglossaries
% \phantomsection\addcontentsline{toc}{section}{\numberline {} DANH MỤC THUẬT NGỮ VÀ TỪ VIẾT TẮT}

\renewcommand\appendixname{PHỤ LỤC}
\renewcommand\appendixpagename{PHỤ LỤC}
\renewcommand\appendixtocname{PHỤ LỤC}

\renewcommand{\figurename}{Hình}
\renewcommand{\tablename}{Bảng}
\renewcommand{\chaptername}{CHAPTER}

% ===================================================


\newpage
\pagenumbering{arabic}

\pagestyle{fancy}
\fancyhf{}
\fancyhead[RE, LO]{\leftmark}
%\fancyhead[LE]{\rightmark}
\fancyfoot[RE, LO]{\thepage}

\chapter{GIỚI THIỆU ĐỀ TÀI}
\subfile{Chapter/1_Introduction} % Phần mở đầu

\newpage
%\pagestyle{fancy} % Áp dụng header và footer
\chapter{CƠ SỞ LÝ THUYẾT}
\label{chapter:Related_works}
\subfile{Chapter/2_Literature_review}


\newpage
%\pagestyle{fancy} % Áp dụng header và footer
\chapter{HÀM SINH SỐ NGẪU NHIÊN CÓ THỂ KIỂM CHỨNG}
\label{chapter:Methodology}
\subfile{Chapter/3_Methodology}

\newpage
%\pagestyle{fancy} % Áp dụng header và footer
\chapter{ĐỀ XUẤT, THỬ NGHIỆM VÀ ĐÁNH GIÁ}
\label{chapter:Experiment}
\subfile{Chapter/4_Theoretical_analysis}

\newpage
%\pagestyle{fancy} % Áp dụng header và footer
% \chapter{ĐÓNG GÓP NỔI BẬT}
% \label{chapter:SolutionAndContribution}
% \subfile{Chapter/5_Numerical_results}

\newpage
\chapter*{KẾT LUẬN VÀ HƯỚNG PHÁT TRIỂN}
\addcontentsline{toc}{chapter}{{\bf  KẾT LUẬN VÀ HƯỚNG PHÁT TRIỂN}\rm}
\label{chapter:conclusion}
\subfile{Chapter/6_Conclusions}

% \newpage
% %\pagestyle{fancy} % Áp dụng header và footer
% \chapter*{SHORT NOTICES ON REFERENCE} %Kết luận và hướng phát triển}
% \label{chapter:reference}
% \subfile{Chapter/7_Reference}

% ===================================================
\newpage
\renewcommand\bibname{REFERENCE}
\printbibliography
\phantomsection\addcontentsline{toc}{chapter}{REFERENCE}

\appendixpage
\appendix
\addappheadtotoc

\titleformat{\chapter}[hang]{\centering\bfseries}{ \thechapter.\ }{0pt}{}[]
\titlespacing*{\chapter}{0pt}{-20pt}{20pt}

\titlecontents{chapter}
    [0.0cm]             % left margin
    {\bfseries\vspace{0.3cm}}                  % above code
    {{\bfseries{\scshape} \thecontentslabel.\ }} % numbered format
    {}         % unnumbered format
    {\titlerule*[0.3pc]{.}\contentspage} 
    
\chapter{SƠ LƯỢC VỀ MẠNG BLOCKCHAIN}
\label{appendix:A}
\subfile{Chapter/Appendix_A}

\newpage
\chapter{KIẾN THỨC TOÁN LIÊN QUAN VÀ KỸ THUẬT GIẢM GAS}
\label{appendix:B}
\subfile{Chapter/Appendix_B}

\end{document}

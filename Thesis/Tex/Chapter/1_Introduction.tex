\documentclass[../main.tex]{subfiles}
\begin{document}
Mọi thứ bắt đầu từ năm 2008 với Bitcoin, mọi người có thể gửi tiền cho bất cứ ai trên thế giới không kể vùng lãnh thổ mà không cần người trung gian. Điều này làm Bitcoin khác biệt so với các hệ thống ngân hàng thế giới. Hơn nữa không ai có thể truy cập được vào tài khoản của bạn. Bước đột phá thứ hai là sự xuất hiện của Ethereum vào năm 2015 (phụ lục \ref{appendix:A}), hệ thống máy tính ảo được xây dựng trên sự đột phá của Bitcoin. Từ đó đã dẫn đến sự phát triển của các ứng dụng và dịch vụ trên mạng lưới phi tập trung (dApps). Các ứng dụng này trước hết sẽ có tính phi tập trung do đặc tính trên mạng phi tập trung, không có cá nhân hay tổ chức nào điều khiển. Tính tất định của máy ảo Ethereum. Tính độc lập, nếu ứng dụng phi tập trung có lỗi logic ở các hợp đồng thông minh thì sẽ không ảnh hưởng tới các chức năng chính của Ethereum. Do đó dApps sẽ có một vài ưu điểm sau:
\begin{enumerate}
    \item Liên tục: Một khi hợp đồng thông minh đã được triển khai trên blockchain thì bất kỳ node nào đều có thể giúp người dùng tương tác với hợp đồng thông minh này.
    \item Riêng tư: Ta không cần phải xác thực danh tính để triển khai các hợp đồng thông minh với các ứng dụng phi tập trung.
    \item Không kiểm duyệt: Không một thực thể đơn lẻ nào có thể chặn bạn gửi transaction, triển khai hợp đồng thông minh, tương tác với hợp đồng thông minh hay là đọc dữ liệu từ blockchain.
    \item Toàn vẹn của dữ liệu: Dữ liệu được đưa lên mạng chuỗi khối là bất biến và không thể chối cãi. 
    \item Tính toán tin cậy: Hợp đồng thông minh sẽ tính toán đúng theo những gì có thể dự đoán mà không cần tin tưởng vào một bên thứ ba.
\end{enumerate}

Tuy nhiên dApps sẽ có một vài mặt hạn chế về sự khó bảo trì, khó nâng cấp. Bởi vì code đã được công khai lên blockchain sẽ khó chỉnh sửa, kể cả là việc tung ra các bản cập nhật vá lỗi hay các rủi ro bảo mật. Chi phí về hiệu suất cũng là một trong những hạn chế. Do cơ chế của mạng phi tập trung, tất cả các transaction đều được xác thực và lan truyền sang xung quang đối với toàn node. Điều này làm cho các transaction sẽ tiêu tốn tài nguyên toàn mạng gấp nhiều lần dù chỉ phức tạp hơn một chút. Đây thực sự là cản trở cũng như là thử thách đối với việc mở rộng ứng dụng phi tập trung.

\section{Hàm sinh số ngẫu nhiên có thể kiểm chứng}
Từ sự phát triển của dApps, sự cần thiết của số ngẫu nhiên là tất yếu.
Số ngẫu nhiên đóng vai trò quan trọng trong lý thuyết trò chơi, với số ngẫu nhiên ta có thể:
\begin{enumerate}
    \item Tạo ra các hộp quà ngẫu nhiên.
    \item Chọn ra người chiến thắng ngẫu nhiên trong một trò chơi xổ số.
    \item Tạo ra các tình huống bất ngờ trong trò chơi.
    \item Tạo lập và phân phối các sản phẩm NFT.
\end{enumerate}
Tuy nhiên việc tạo ra số ngẫu nhiên trên ethereum không giống như trên máy tính thông thường bởi chính tính minh bạch, phân tán của nó. Nếu ta triển khai bất kỳ thuật toán giả ngẫu nhiên nào trên hợp đồng thông minh thì tại cùng một thời điểm bất cứ ai với bất cứ node nào đều cho ra cùng một kết quả.

Việc giải quyết được bài toán sinh số ngẫu nhiên trên mạng blockchain sẽ tăng tính bảo mật cho các dự án phi tập trung, góp phần làm tăng trải nghiệm người dùng, dẫn ra một hướng đi mới của cơ chế đồng thuận cho mạng blockchain - bằng chứng cổ phần (proof of stake) \cite{chen2016algorand}.

\section{Mục tiêu và phạm vi đề tài}
Hiện nay với đa số các ứng dụng phi tập trung nhỏ có sử dụng yếu tố ngẫu nhiên, đội ngũ phát triển sẽ lấy nguồn ngẫu nhiên từ blochhash do cơ sở dữ liệu chuỗi khối này cứ tiếp tục được nối dài và các giá trị blockhash (thêm chi tiết về blockhash phụ lục \ref{appendix:A}) là khó bị thao túng. Tuy nhiên các thợ đào (miner) hoàn toàn có thể bỏ đi lợi ích của việc công khai một block đúng để lấy lợi ích từ việc thao túng nguồn ngẫu nhiên này. 

Giải pháp mang tính đột phá hiện nay đó là Chain Link \cite{chainlink}. Dự án cho phép trả về số ngẫu nhiên kèm theo bằng chứng. Dự án được xây dựng từ thuật toán VRF. Tuy nhiên sẽ không có gì đảm bảo cho việc oracle của ChainLink không bị sập.

Cũng dựa trên thuật toán sinh số ngẫu nhiên có thể kiểm chứng, mạng blockchain Algorand đã xây dựng cơ chế đồng thuật bằng chứng cổ phần (PPOS - pure proof of stake) \cite{chen2016algorand}. Cơ chế PPOS giúp cho giao thức đồng thuận tiêu tốn tối thiểu công tính toán ở các node, giải quyết vấn đề về môi trường trong việc vận hành mạng blockchain với bằng chứng công việc (POW- proof of work). 

Trong phạm vi đồ án này em sẽ đứng trên vai của ChainLink và đưa ra giải pháp cho định hướng thiết kế mới giúp bất kỳ ai cũng có thể cung cấp dịch vụ sinh số ngẫu nhiên trên blockchain.
\section{Định hướng giải pháp}
Từ định hướng em đã trình bày ở trên em sẽ sử dụng thuật toán sinh số ngẫu nhiên có thể kiểm chứng được trình bày và chứng minh tính đúng đắn tại \cite{papadopoulos2017making}. Về ý tưởng thuật toán khá tương đồng với ý tưởng chữ ký số. Tức là ai cũng có thể xác thực được chữ ký số hợp lệ nhưng chỉ có một người sở hữu khóa bí mật mới có thể ký được lên transaction hợp lệ. 

Với thuật toán này em có thể giải quyết được vấn đề không ai có thể thao túng được nguồn ngẫu nhiên tới hợp đồng thông minh. 

Em sẽ thiết kế kiến trúc mô hình gửi và nhận số ngẫu nhiên trên hợp đồng thông minh sử dụng mạng lưới Oracle. Xây dựng mạng oracle đơn giản để tương tác với hợp đồng thông minh.
\section{Bố cục đồ án}
Phần còn lại của đồ án em xin được tổ chức theo bố cục như sau:

Chương \ref{chapter:Related_works} trình bày về cơ sở lý thuyết và các kiến thức toán quan trọng, nền tảng em đã học được trên ghế đại học.

Chương \ref{chapter:Methodology} em trình bày về thuật toán sinh số ngẫu nhiên có thể kiểm chứng, thuật toán và các tính chất bảo mật của thuật toán.

Chương \ref{chapter:Experiment} trình bày về đề xuất mô hình, thử nghiệm và đánh giá kết quả mô hình trong thực tế.

Phần kết luận em kết lại những gì đạt được trong quá trình đồ án và nêu ra định hướng phát triển của đồ án trong tương lai.
\end{document}



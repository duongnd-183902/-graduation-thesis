\documentclass[../main.tex]{subfiles}
\begin{document}

\begin{center}
    \Large{\textbf{TÓM TẮT NỘI DUNG ĐỒ ÁN}}\\
\end{center}
\vspace{1cm}
% Sinh viên viết tóm tắt ĐATN của mình trong mục này, với 200 đến 350 từ. Theo trình tự, các nội dung tóm tắt cần có: (i) Giới thiệu vấn đề (tại sao có vấn đề đó, hiện tại được giải quyết chưa, có những hướng tiếp cận nào, các hướng này giải quyết như thế nào, hạn chế là gì), (ii) Hướng tiếp cận sinh viên lựa chọn là gì, vì sao chọn hướng đó, (iii) Tổng quan giải pháp của sinh viên theo hướng tiếp cận đã chọn, và (iv) Đóng góp chính của ĐATN là gì, kết quả đạt được sau cùng là gì. Sinh viên cần viết thành đoạn văn, không được viết ý hoặc gạch đầu dòng.

Cùng với thống kê, lý thuyết xác suất là một nhánh của toán học đã được phát triển để giải quyết với sự không chắc chắn. Lý thuyết toán học cổ điển đã thành công trong việc mô tả thế giới như một chuỗi các sự kiện cố định và thực tế có thể quan sát được, tuy nhiên đến thế kỷ XVII, nó phần lớn không đủ khả năng giải quyết các quá trình hoặc thí nghiệm liên quan đến các kết quả không chắc chắn hoặc ngẫu nhiên. Ban đầu được thúc đẩy bởi mong muốn phân tích các trò chơi cờ bạc của nhà toán học và sau đó là phân tích khoa học về các ca tử vong trong ngành y, lý thuyết xác suất đã được phát triển như một công cụ khoa học giải quyết các mô tả liên quan tới sự ngẫu nhiên.

Với Ethereum blockchain, việc tạo ra các số ngẫu nhiên hay giả ngẫu nhiên gặp nhiều khó khăn hơn máy tính và hệ thống thông thường bởi tính công khai, tất định và phân tán của máy ảo EVM. Hãy tưởng tượng, bạn tham gia một trò chơi có thưởng với mức phí tham gia là 1 ETH. Giải thưởng sẽ là 1000 ETH. Kết quả thắng cuộc được ban tổ chức nói rằng dựa trên một số ngẫu nhiên. Vậy làm sao để chắc rằng con số này không bị thao túng bởi một bên thứ ba hay con số này thực sự là biến ngẫu nhiên có phân phối đều. Hơn nữa nếu bạn là chủ một chương trình game deFi và các vật phẩm trong game đều yêu cầu một số ngẫu nhiên, vậy tốc độ sinh ra số ngẫu nhiên có đủ tốt hay không. 

% Cách tiếp cận thứ nhất: sử dụng blockhash của block hiện tại. Như ta đã biết thì blockchain liên tục phát triển, và blockhash luôn luôn thay đổi và được chấp nhận qua cơ chế đồng thuận. Do đó ta sẽ dùng blockhash của block hiện tại để làm hạt giống cho thuật toán sinh số ngẫu nhiên. Tuy vậy một thợ đào (miner) sẽ quyết định có công bố khối họ làm được ra mạng hay không. Cách tiếp cận thứ hai: ta sẽ sử dụng một API bên ngoài. Đó là một ý tưởng thật tệ. Bởi vì nguyên tắc tạo ra blockchain là phi tập trung, ta không nên phụ thuộc vào một nguồn bên ngoài như các API random.org. Sẽ ra sao nếu tin tặc thao túng API này và luôn trả về kết quả mặt thứ sáu của một con xúc sắc trong hợp đồng thông minh của ta.


Cách tiếp cận đồ án: sử dụng hàm sinh số ngẫu nhiên có thể kiểm chứng. Trong mật mã học, một hàm ngẫu nhiên có thể kiểm chứng (VRF) là một hàm khóa công khai giả ngẫu nhiên mà cung cấp bằng chứng rằng kết quả được tính toán đúng. Chủ sở hữu khóa bí mật có thể tính toán kết quả trả về của hàm cũng như bằng chứng với bất kỳ đầu vào nào. Mọi người khác có thể kiểm chứng kết quả trả về thực sự được tính toán đúng thông qua bằng chứng và khóa công khai.
\end{document}
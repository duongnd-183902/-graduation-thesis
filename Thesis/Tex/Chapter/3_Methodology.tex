\documentclass[../main.tex]{subfiles}
\begin{document}
Trong phần \ref{chapter:Methodology} đồ án trình bày về hàm sinh số ngẫu nhiên có thể kiểm chứng được, đây là thuật toán cốt lõi cho việc triển khai mô hình giải quyết bài toán sinh số ngẫu nhiên trên mạng chuỗi khối có thể kiểm chứng. Đi theo đó. chương trình bày chứng minh tóm tắt các tính chất của VRF.

% Một phần kiến thức quan trong khác tuy không phải là cơ sở xây dựng lên thuật toán VRF nhưng lại là môi trường để em triển khai thử nghiệm và đánh giá kết quả thuật toán đó là mạng blockchain ethereum. Em xin được tóm gọn những kiến thức cần thiết và những yếu tố có tác động to lớn tới thiết kế kiến trúc trong chương \ref{chapter:Experiment} tại phụ lục \ref{appendix:A}. Kể từ đây, nếu không nói gì thêm thì khi em nói tới mạng blockchain thì có nghĩa em đang nói tới mạng blockchain ethereum. 

\section{Định nghĩa hàm VRF}
VRF là bộ ba thuật toán có độ phức tạp thời gian đa thức đó là:
\begin{enumerate}
    \item Thuật toán sinh khóa $G$ nhận đầu vào là một chuỗi bytes và trả về kết quả hai chuỗi bytes (khóa công khai $pk$ và khóa bí mật $sk$).
    \item $F = (F_1,F_2)$ nhận đầu vào là hai chuỗi bytes (khóa bí mật $sk$ và $x$) trả về hai chuỗi bytes $v =F_1(sk,x)$ và bằng chứng tương ứng $\text{proof} = F_2(sk,x)$.
    \item $V$ - hàm xác thực với bốn tham số đầu vào là $pk,x,v,\text{proof}$. Trả về kết quả hợp lệ hoặc không.
\end{enumerate}


\section{Thuật toán}

Một hàm ngẫu nhiên có thể xác thực (VRF) là một hàm khóa công khai giả ngẫu nhiên mà cung cấp bằng chứng rằng kết quả được tính toán đúng. Chủ sở hữu khóa bí mật có thể tính toán kết quả trả về của hàm cũng như bằng chứng với bất kỳ đầu vào nào. Mọi người khác có thể kiểm chứng kết quả trả về thực sự được tính toán đúng thông qua bằng chứng và khóa công khai.

Trong phần này, em sẽ trình bày về bộ ba thuật toán chính của hàm sinh số ngẫu nhiên có thể kiểm chứng được. Chi tiết thuật toán, giao thức, các tính chất được trình bày trong bài báo  \cite{papadopoulos2017making} "Making NSEC5 practical for DNSSEC".

\textbf{Các tham số công khai:} $q$ là số nguyên tố, $G$ là nhóm cyclic cấp $q$ với phần tử sinh $g$. Bởi vì tính kiểm tra các phần tử thuộc $G$ sẽ tốn nhiều công tính toán, ta giả sử $G$ là một nhóm con của $E$ sao cho việc kiểm tra phần tử trong $E$ là dễ hơn, hệ số $f = |E|/|G|$ không là chia hết $q$. Dĩ nhiên $G$ có thể trùng với $E$, trong trường hợp này hệ số $f = 1$. 

Ta giả thiết rằng $q,g,f,G,E$ là các tham số công khai.

$H_1$ là hàm băm (random oracle) ánh xạ một chuỗi có độ dài tùy ý vào $G-\{1\}$. $H_2$ là một hàm lấy biểu diễn của một phần tử của $E$ và rút gọn nó về độ dài thích hợp. $H_3$ là một hàm băm (random oracle) ánh xạ một đầu vào với độ dài tùy ý tới số nguyên $l$-bit.

\textbf{Thuật toán sinh khóa (Keys)} Khóa bí mật $x \in \{1,2,...,q-1\}$ được chọn ngẫu nhiên phân phối đều. Khóa công khai của VRF là $PK = g^x$.

\textbf{Thuật toán sinh bằng chứng (Hashing)} Cho khóa bí mật $x$ và đầu vào $\alpha$, tính toán bằng chứng $\pi$ theo:
\begin{enumerate}
    \item Tính phần tử $h = H_1(\alpha)$ và $\gamma = h^x$.
    \item Chọn một số ngẫu nhiên dùng một lần $k \in \{1,2,...,q-`1\}$.
    \item Tính $c = H_3(g,h,g^x,h^x,g^k,h^k)$.
    \item $s = k-cx\pmod q$ 
\end{enumerate}
Bằng chứng $\pi$ là nhóm các phẩn tử gồm $\gamma$, $c$, $s$. Kết quả trả về của VRF $\beta= F_{SK}(\alpha)$. Do đó
$$\pi =(\gamma,c,s)$$
$$\beta = H_2(\gamma)$$

\textbf{Thuật toán kiểm chứng (Verifying)} Cho khóa công khai $PK$, xác thực bằng chứng $\pi = (\gamma,c,s)$ tương ứng với đầu vào $\alpha$ kết quả $\beta$
\begin{enumerate}
    \item Tính $u = (PK)^c \cdot g^s$ \\
    Nếu mọi thứ đúng thì $u = g^k$.
    \item Cho đầu vào $\alpha$, ta thu được $h = H_1(\alpha)$. Kiểm tra $\gamma \in G$. Tính $v = (\gamma)^c\cdot h^s$ \\
    Nếu mọi thứ đúng thì $v = h^k$.
    \item Kiểm tra $c = H_3(g,h,PK,\gamma,u,v)$
\end{enumerate}
Cuối cùng, tính $\beta = H_2 (\gamma)$.



\section{Tính bảo mật của VRF}
VRF có ba tính chất: tính duy nhất (uniqueness), tính giả ngẫu nhiên (Pseudorandomness), tính kháng xung đột (collistion - resistance).
Để đơn giản ta giả sử $E = G$ do đó $f =1$.

\textbf{Tính duy nhất (Uniqueness)} (\hspace{1sp}\cite{chainlink}) Ta sẽ chứng minh phản chứng. Giả sử một ai đó biết được khóa bí mật $x$, cùng với đầu vào $\alpha$, họ đưa ra kết quả gian lận $\beta_1\neq H_2([H_1(\alpha)]^x)$. Một bằng chứng hợp lệ $\pi_1 = (\gamma_1,c_1,s_1)$ cho kết quả $\beta_1$. Các hàm xác thực sẽ tính toán $h = H_1(\alpha)$ và 
$$ u = (g^x)^{c_1}g^{s_1}
$$
$$ v = (\gamma_1)^{c_1}h^{s_1}
$$
Lấy logarithm của phương trình thứ nhất cơ số $g$ và phương trình thứ hai cơ số $h$. Trừ hai kết quả ta được biểu diễn của $c_1$
$$c_1 \equiv \frac{\log_g(u) - \log_h(v)}{x-\log_h(\gamma_1)}\mod q
$$
Do $\gamma_1 \neq h^x$ (do $\beta_1$ không phải là kết quả đúng), do đó mẫu số khác không, có đúng một số $c_1$ modulo $q$ sao cho thỏa mãn với $(g,h,g^x,\gamma,u,v)$ và $s$ bất kỳ. Do $H_3$ là một hàm ngẫu nhiên, kết quả trả ra cũng là ngẫu nhiên. Vậy xác suất để nó bằng với một số được xác định trước trong đẳng thức trên là không đáng kể. 

\textbf{Tính giả ngẫu nhiên (Pseudorandomness)} (\hspace{1sp}\cite{chainlink}) Ta đã biết khóa bí mật VRF $x$ sẽ rất khó bị tìm ra bởi bất cứ ai. Các tin tặc sẽ biết khóa công khai, $g$, $g^x$ và dễ dàng tính được $h = H_1(\alpha)$ với mọi đầu vào $\alpha$. Theo giả định Diffie – Hellman (DDH) quyết định $h^x$ trông sẽ như là ngẫu nhiên khi biết $(g,g^x,h)$. Do đó $H_2(h^x)$ là giả ngẫu nhiên trong phạm vi của $H_2$.

\textbf{Tính kháng xung đột (Collision resistance)} (\hspace{1sp}\cite{chainlink}) Sự xung đột xảy ra khi $H_2(h_1^x) = H_2(h_2^x)$ trong đó $h_1 = H_1(\alpha_1)$ và $h_2 = H_1(\alpha_2)$ với $\alpha_1 \neq \alpha_2$. 
Giả sử rằng $H_2$ là một hàm $\tau$-to-1. Tức là với mỗi giá trị $h_1$ sẽ có tối đa $\tau$ giá trị $h_2$ có thể gây ra xung đột. Từ việc $h_1$ và $h_2$ thu được từ giả thiết ngẫu nhiên mạnh (random oracle queries). Một cặp như vậy sẽ khó gây ra xung đột sau $Q_H$ yêu cầu của $H_1$ miễn là $G$ lớn hơn $\tau Q_h^2 /2$.


\section{VRF với đường cong elliptic Secp256k1}
Trong đồ án này em sử dụng hệ mật đường cong elliptic (EC) thay vì hệ mật RSA bởi vì khi triển khai trên EC em sẽ thu được bằng chứng VRF có kích cỡ nhỏ hơn trong khi đạt được cùng mức bảo mật so với bằng chứng VRF được triển khai trên RSA (chi tiết \cite{papadopoulos2017making}).

Trong đồ án này em sử dụng đường cong eliptic secp256k1 bởi vì hai nguyên nhân.
\begin{enumerate}
    \item Đường cong secp256k1 là một nhóm cyclic. \ref{appendix:secp256k1}
    \item Đây là đường cong được dùng cho chữ ký số trên ethereum.
\end{enumerate}
Với việc tận dụng được các hàm chữ ký số trên ethereum sẽ giảm được chi phí gas khi triển khai cũng như giảm thời gian xác thực bằng chứng ngẫu nhiên trên mạng blockchain. 

Các tham số chi tiết về đường cong em xin được trình bày trong phần phụ lục \ref{appendix:secp256k1}.

\section{Các hàm băm}
Trong thuật toán trong phần trước ta có sử dụng một hàm băm $H_1$ ánh xạ một chuỗi có độ dài tùy ý tới một điểm trong đường cong eliptic. Có một kỹ thuật mang ý tưởng thử và loại được nhắc đến trong \cite{icart2009hash}. Giả sử ta có phương trình đường cong $y^2 = x^3 +ax+b$ có bậc $qf$. Với một input là $\alpha$. Ta sẽ ghép đầu vào với một biến đếm $i$, sau đó cho qua hàm băm SHA256 được gái trị lưu vào $h$. Bước tiếp theo là kiểm tra $h^3+ax+b$ có là phần dư bậc hai hay không. Nếu thỏa mãn thì $h$ là một tọa độ $x$ hợp lệ trên đường cong. Nếu không, ta tăng biến đếm $i$ lên một đơn vị và thử lại cho đến khi thu được kết quả. Sau cùng ta tăng kết quả mũ $f$ để được phần tử trong $G$. 
Quá trình này có kỳ vọng với hai lần thử ta sẽ thu được kết quả thông qua bổ đề \ref{bd:quadraticResidue}.

Trong triển khai thực tế em vẫn sử dụng ý tưởng thử và loại tuy nhiên có một chút khác biệt thay vì em ghép input với một biến chạy đến khi thu được kết quả thì em băm lại kết quả vừa loại để được một điểm thử mới.

Bắt đầu bằng thuật toán băm từ một bytes tới một điểm trên trên trường hữu hạn $F_p$. 

\textbf{Thuật toán fieldHash} Với đầu vào là bytes $b$.
\begin{enumerate}
    \item $x$ = Hash($b$).
    \item Kiểm tra $x < p$, nếu không $x =\text{Hash}(x)$. Quay lại bước 2.
    \item Trả về $x$.
\end{enumerate}

Tiếp theo là thuật toán tính một điểm ứng cử viên thuộc đường cong Secp256k1

\textbf{Thuật toán newCandidateSecp256k1Point} Với đầu vào là bytes $b$
\begin{enumerate}
    \item Tính $px = \text{fieldHash}(b)$
    \item Tính $\text{ySquare} = px^3 + 7 \mod{p}$
    \item Tính $py = \text{ySquare}^{\text{SQRT\_POWER}}$
    \begin{description}
    \item Với $\text{SQRT\_POWER} = \frac{(p+1)}{4}$
    \end{description}
    \item Kiểm tra  nếu $py \% 2 = 1$ thì $py = p - py$.
    \item Trả về $[px,py]$
\end{enumerate}

Khi ấy ta chỉ cần kiểm tra điểm ứng cử viên, nếu thỏa mãn phương trình đường cong Elliptic Secp256k1. Kết thúc thuật toán, nếu chưa thỏa mãn tiếp tục lấy kết quả đầu ra trước làm đầu vào cho thuật toán điểm ứng cử viên và kiểm tra điều kiện.

\textbf{Thuật toán hashToCurve} Với đầu vào là một bytes $b$
\begin{enumerate}
    \item $rv = \text{newCandidateSecp256k1Point}(b)$
    \item Kiểm tra $rv$ là điểm trên đường cong. Nếu không $b = \text{bytes(}[rv[0])$ quay lại bước 1.
\end{enumerate}

% Với các hàm băm $H_2$ và hàm băm $H_3$ ta có thể đơn giản sử dụng thuật toán băm như Keecak256 hay Sha256.
\end{document}
\documentclass[../main.tex]{subfiles}
\begin{document}
Trong phần này em trình bày về các kiến thức toán học quan trọng có liên quan tới kỹ thuật làm giảm chi phí gas trong khi triển khai hợp đồng thông minh phục vụ cho giải quyết bài toán đặt ra ban đầu. Em sẽ trình bày một vài phương án tiếp cận một vấn đề đặt ra, việc này sẽ phục vụ cho việc phân tích, lựa chọn công nghệ sử dụng của em tại chương \ref{chapter:Experiment}.
\section{Các tính chất của VRF}


\section{Đường cong Elliptic Secp256k1}
\label{appendix:secp256k1}
Hệ các điểm thỏa mãn phương trình Weierstrass trên trường $F_p$

Phương trình Weierstrass:
\begin{equation*}
    y^2 = x^3 + ax +b
\end{equation*}

\begin{table}[h!]
    \centering
    
    \begin{tabular}{||c l||}
    \hline
    Tham số       & Giá trị   \\
    \hline \hline
    $p$     &   \begin{tabular}{c}
        \texttt{0xFFFFFFFFFFFFFFFFFFFFFFFFFFFFFFFF}\\
        \texttt{FFFFFFFFFFFFFFFFFFFFFFFEFFFFFC2F}
    \end{tabular} \\
    $a$     & \begin{tabular}{c}
        \texttt{0x0} 
    \end{tabular}\\
    $b$     & \begin{tabular}{c}
        \texttt{0x7} 
    \end{tabular}\\
    $n$    & \begin{tabular}{c}
        \texttt{0xFFFFFFFFFFFFFFFFFFFFFFFFFFFFFFFE}\\
        \texttt{BAAEDCE6AF48A03BBFD25E8CD0364141}
    \end{tabular}\\
    
    $Gx$    & \begin{tabular}{c}
        \texttt{0x79BE667EF9DCBBAC55A06295CE870B07}\\
        \texttt{029BFCDB2DCE28D959F2815B16F81798}
    \end{tabular}\\
    
    $Gy$    & \begin{tabular}{c}
        \texttt{0x483ADA7726A3C4655DA4FBFC0E1108A8}\\
        \texttt{FD17B448A68554199C47D08FFB10D4B8}
    \end{tabular}\\
    $f$     & \begin{tabular}{c}
        \texttt{1} 
    \end{tabular}\\
    \hline
    \end{tabular}
    \caption{Bảng tham số của đường cong secp256k1}
    \label{table:secp256k1_params}
\end{table}

Secp256k1 là một nhóm giao hoán với phép tính cộng có bậc là số nguyên tố. Do đó theo bổ đề \ref{bd:primeGroup} secp256k1 là một nhóm cyclic với phép tính cộng.

% $a =$ 0x1\\
% $b =$ 0x7\\
% $G_x =$ 0x79BE667EF9DCBBAC55A06295CE870B07029BFCDB2DCE28D959F2815B16F81798\\
% $G_y =$ 0x483ADA7726A3C4655DA4FBFC0E1108A8FD17B448A68554199C47D08FFB10D4B8

\section{ECDSA và giả chữ ký}
 ECDSA (Elliptic Curve Digital Signature Algorithm), là một lược đồ chữ ký dựa trên mật mã đường cong elliptic (ECC). ECDSA dựa vào phép toán của các nhóm cyclic và đường cong elliptic trên các trường hữu hạn. Các khóa và chữ ký ECDSA ngắn hơn trong RSA cho cùng mức độ bảo mật. Chữ ký ECDSA 256-bit có cùng độ bền bảo mật như chữ ký RSA 3072-bit.

Trong phần này em đưa ra một kỹ thuật giả chữ ký tận dụng precompile của ethereum để tiết kiệm gas khi triển khai thực tế. Trước hết em sẽ đi từ các thuật toán ký và xác thực chữ ký trên đường cong elliptic. Việc này là cơ sở cần thiết cho tính đúng đắn thuật toán ecrecover.


\begin{table}[h!]
    \centering
    \begin{tabular}{||c l||}
    \hline
    Tham số       & Ý nghĩa   \\
    \hline \hline
    CURVE       & Trường hữu hạn đường cong sử dụng \\
    $G$         & Phần tử sinh của đường cong \\
    $n$         & Bậc của đường cong \\
    $sk$        & Khóa bí mật \\
    $pk$        & Khóa công khai $sk\times G$\\
    $m$         & message\\
    $L_n$       & Độ dài bit của bậc $n$\\
    $Hash$      & Thuật toán hash lấy $L_n$ bit\\
    \hline
    \end{tabular}
    \caption{Bảng tham số ECDSA}
    \label{table:ECDSA_params}
\end{table}

Để cho đơn giản, trong bảng tham số bảng \ref{table:ECDSA_params} ta giả sử $n$ là một số nguyên tố. Trong thực tế ta có thể đạt được điều này nhờ việc xét nhóm con của nhóm ban đầu để thu được một nhóm cyclic.

\textbf{ECDSA Signature Generation} (\cite{johnson2001elliptic})


Để ký thông điệp $m$, với các tham số trong bảng \ref{table:ECDSA_params}.
\begin{enumerate}
    \item Sinh số ngẫu nhiên $k \in [1,n-1]$.
        \begin{description}
        \item Trong ECDSA tất định như ethereum thì $k$ sẽ được tính từ $sk$ và $h$ (\cite{pornin2013rfc}).
        \end{description}
    \item Tính $(x_1, y_1) = k \times G$ và $r = x_1 \mod n$. Nếu $r = 0$ quay lại bước 1.
    \item Tính $k^{-1} \mod n$
    \item Tính $e = Hash(m)$
    \item Tính toán bằng chứng chữ ký $s = k^{-1}\times (e+r\times sk)\mod n$.
    \item Signature($r,s$).
\end{enumerate}

\textbf{ECDSA Verify Signature} (\cite{johnson2001elliptic})


Để xác thực chữ ký $(r,s)$ lên thông điệp $m$, với các tham số trong bảng \ref{table:ECDSA_params}.
\begin{enumerate}
    \item $e = Hash(m)$.
    \item Tính: $w = s^{-1} \mod{n}$.
    \item Tính $u_1 = e\times w\mod n$ và $u_2 = r\times w\mod n$.
    \item Tính $X = (x_1,y_1) = u_1\times G + u_2 \times pk$. Nếu $X = \mathcal{O}$ từ chối chữ ký. Ngược lại $v = x_1$.
    \item Chấp nhận chữ ký khi và chỉ khi $v = r$.
\end{enumerate}

Nếu chữ ký $(r,s)$ lên thông điệp $m$ là hợp lệ thì với $s = k^{-1}(e + sk \times r)$
\begin{align*}
    k   &\equiv s^{-1}(e+sk \times r)\\
        &\equiv s^{-1} \times e + s^{-1}\times r\times sk\\
        &\equiv w\times e + w \times r \times sk\\
        &\equiv u_1 + u_2 \times sk \mod n\\
\end{align*}

Do đó $u_1 \times G + u_2 \times pk = (u_1 + u_2 \times sk)\times G$ và $v = r$.

\textbf{ECDSA Public Key Recovery From Signature}
Với thông điệp $m$, chữ ký $(r,s)$ tính khóa công khai $pk$.
\begin{enumerate}
    \item Kiểm tra $(r,s) \in [1,n-1]$. Nếu không, chữ ký không hợp lệ.
    \item Tính $R = (x_1,y_1)$ với $x_1 = r$, $y_1$ là giá trị thỏa mãn đường cong elliptic.
    \item $e = hash(m)$.
    \item Tính $u_1 = -e \times r^{-1}\mod n$ và $u_2 = s\times r^{-1}\mod{n}$.
    \item Tính $X = u_1 \times G + u_2 \times R$.
    \item Trả về $X$
\end{enumerate}
Ta sẽ chứng minh tính đúng đắn của thuật toán bắt đầu từ bước tính $X =u_1 \times G + u_2 \times R$.
\begin{align*}
    X   &= u_1 \times G + u_2 \times R \\
        &= u_1 \times G + u_2 \times k \times G\\
        &= (u_1 + u_2 \times k) \times G\\
        &= (-e\times r^{-1} + s\times k \times r^{-1}) \times G\\
        &= (-e\times r^{-1} +  k^{-1}(e + sk \times r) \times k \times r^{-1}) \times G\\
        &= (-e\times r^{-1} + (e\times r^{-1} + sk)) \times G\\
        &= sk \times G\\
        &= pk
\end{align*}

Cụ thể trong ethereum, hàm ecrecover sẽ trả về kết quả là address tương ứng với khóa công khai trên đường cong secp256k1.

Hàm ecrecover là một trong những precompiled của ethereum (\cite{wood2014ethereum}). Điều này làm tiết kiệm chi phí gas đáng kể so với cách triển khai hàm ecrecover trên một hợp đồng thông minh khác, bất kế là cùng thuật toán với cài đặt của ethereum. 

Tiếp theo em xin trình bày về kỹ thuật giả chữ ký. Kỹ thuật này sẽ tận dụng hàm ecrecover để tính toán tổ hợp tuyến tính hai điểm trên đường cong mà trong đó có một điểm là phần tử sinh của đường cong.

Cụ thể tại bước thứ năm của thuật toán \textbf{ECDSA Public Key Recovery From Signature} ta có được
\begin{equation*}
    X = u_1 \times G + u_2 \times R
\end{equation*}
Mục tiêu của ta là tính tổ hợp tuyến tính $u = s \times G + c \times pk $ như bước thứ nhất của thuật toán \textbf{Verifying}. Từ thuật toán precompiled $\text{ecrecover}(hash, v, r, s)$ (\cite{wood2014ethereum}) với $hash$ là kết quả băm của thông điệp $m$. Tham số $v$ giúp chỉ rõ điểm $R = (x_1,y_1), x_1 = r$ nằm trên mặt phẳng tọa độ. $r,s$ như trên thuật toán đã trình bày. Ta đưa ra giả chữ ký:

\begin{enumerate}
    \item $\text{pseudoHash} = -s\times pk[0]$
    \item $r = pk[0]$
    \item $v =(pk[1]\%2 ==0)?27:28$
    \item $\text{pseudoSignature} = c \times pk[0]$
    \item $\text{ecrecover}(\text{pseudoHash}, v, r, \text{pseudoSignature}) = c\times pk + s \times G$
\end{enumerate}
Thật vậy,
\begin{align*}
    u   &= u_1 \times G + u_2 \times R\\
        &= u_1 \times G + u_2 \times pk\\
        &= s\times pk[0] \times pk[0]^{-1} + c\times pk[0] \times pk[0]^{-1} \times pk\\
        &= s \times G + c \times pk\\
\end{align*}

Tương tự kỹ thuật giả chữ ký tính tổ hợp tuyến tính giữa hai điểm trên đường cong. Ta có thể tận dụng kỹ thuật này để xác thực tích vô hướng với một điểm bằng cách đặt tham số $s$ bằng $0$.
\begin{equation*}
    \text{scalar} \times \text{multiplicand} = \text{ecrecover}(0,v,\text{multiplicand}[0],\text{scalar} \times \text{multiplicand}[0])
\end{equation*}
Với tham số $v$ giúp chỉ rõ điểm $R = (x_1,y_1), x_1 = r$ nằm trên mặt phẳng tọa độ.
\section{Bổ đề}
\begin{bd}
\label{bd:_primeGroup}
Cho $(G,\cdot)$ là một nhóm giao hoán, $x \in G$, $a,b \in \mathbb{Z}$ nguyên tố cùng nhau. Nếu $x^a = x^b = 1$ thì $x=1$.
\end{bd}

Ta sẽ chứng minh bổ đề bằng bổ đề của Bezout. Tồn tại $k,l \in \mathbb{Z}$ sao cho $ak+bl = 1$. Ta có:
\begin{equation*}
    x = x^{ak+bl} = x^{ak}\cdot x^{bl} = 1^k \cdot 1^l = 1
\end{equation*}

\begin{bd}
\label{bd:primeGroup}
Mọi nhóm giao hoán có bậc là số nguyên tố thì là nhóm cyclic.
\end{bd}
Gọi $P$ là nhóm giao hoán bất kỳ có bậc là số nguyên tố $p$. Với mọi $x \in P$ và $x \neq 1$ ta xét
\begin{equation*}
    S = \{1,x,x^2,...,x^{p-1}\} \subseteq P
\end{equation*}

Không mất tính tổng quát, giả sử rằng có hai phần tử $x^u = x^v$ với $u<v$. Do đó $x^{u-v} =1$ hơn nữa $1 \leq u-v \leq p-1$, $u-v$ và $p$ nguyên tố cùng nhau. Theo bổ đề \ref{bd:_primeGroup} $x^{u-v} = x^p = 1$ ngụ ý rằng $x=1$. Mâu thuẫn. Do vậy mọi phần tử của $S$ là khác nhau. Mặt khác $|S| = P$ do đó $S = P$ và $P$ là nhóm cyclic.

\begin{bd}
\label{bd:quadraticResidue}
Cho $p$ là số nguyên tố lẻ. Ta sẽ có $\frac{p-1}{2}$ phần dư bậc hai và $\frac{p-1}{2}$ phần tử không phải là phần dư bậc hai trong khoảng từ $\{1,...,p-1\}$.
\end{bd}
Ta có $k$ và $-k = p - k$ có chung bình phương khi mod $p$. Có nghĩa là 1 và $p-1$ có chung bình phương, 2 và $p-2$ có chung bình phương và $\frac{p-1}{2}$ và $\frac{p-1}{2}+1$ có chung bình phương khi mod $p$.

Do đó, ta sẽ có số bình phương khác nhau là $\frac{p-1}{2}$ - số các số là phần dư bậc hai và $\frac{p-1}{2}$ số khác không phải là phần dư bậc hai. 
\end{document}